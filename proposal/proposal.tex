	\documentclass[a4paper,11pt, french]{article}
\usepackage[english]{babel}
\usepackage[utf8]{inputenc}
\usepackage{multicol}
\usepackage{graphicx}
\usepackage{float} 
\usepackage{fullpage}
\usepackage{algorithm}
\usepackage{algorithmic}
\usepackage{titlesec}
\usepackage{caption}
\usepackage{subcaption}
%\usepackage{titlesec}
\usepackage{tikz}

\usepackage{graphicx}
\usepackage{caption}
\usepackage{subcaption}
\usetikzlibrary{arrows}
\usepackage{bm}
\usepackage{latexsym}
\usepackage{amsmath}
\bibliographystyle{plain}

\usepackage{hyperref}
\renewcommand{\thesubsection}{Stage \Alph{subsection}}
\renewcommand{\thesubsubsection}{Q\Roman{section}.\Alph{subsection}.\arabic{subsubsection}}
\renewcommand{\thesection}{Part \arabic{section}}


\begin{document}

\title{Project proposal}
\author{Judith Abécassis, Timothée Lacroix, Arthur Mensch} 
\date{\today}
\maketitle

\section*{Investigation plan for our project}

We have chosen to work on the "Crowd counting with convolutional neural networks" topic, based on the previous work presented in \cite{basepaper}, \cite{multisource}. We are planning to experiment several variants of what was tested in both these papers and to benchmark their performances on the dataset used in \cite{multisource}, that contains 50 crowd images containing 64K annotated humans. Here are the variants we plan to test :
\begin{enumerate}
\item train regressor from \cite{basepaper} with features from \cite{basepaper} and regressor from \cite{basepaper} with features from \cite{multisource} on this benchmarking dataset to get the baseline performances;
\item change the features and use Overfeat features using \cite{overfeat};
\item explore some possible improvements by the MRF smoothing technique from \cite{multisource} on the different features mentionned.
\end{enumerate}

\section*{Sharing of the work between members}
Here is how we plan to divide the work between our three team members:
\begin{description}
\item[Person 1] Train regressor on various layers of Overfeat
\item[Person 2] Get baseline values with features from \cite{basepaper} and \cite{multisource}
\item[Person 3] Explore the improvements yielded by MRF smoothing
\end{description}


\section*{Evaluated metrics}
We will evaluate various improvements by using different metrics. We will evaluate each method with and without smoothing to assess its performance gain.
\begin{description}
\item Cross-Validated Mean Absolute Error on the cells dataset (used for regressor training) and Mean Absolute Error on the crowd dataset.
\item Normalized Mean Absolute Error, to account for higher errors in more crowded pictures.
\item Absolute Error and Normalized Absolute Error averaged on groups of five images, grouped by total groundtruth count.
\end{description}
This will allow us to precisely compare our results to those presented in \cite{basepaper} and \cite{multisource}.

\bibliography{biblio}

\end{document}